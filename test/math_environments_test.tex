\documentclass[lang=ngerman, margins=small, extendmath=true]{scrhmwrk}

\course{ADS – Übungen}
\tutor[Übungsleiterin]{Malin Rau \& Maren Kaluza}
\group{3 \& 8}
\participant{Wolfgang Ramos \& Ole Numssen}
\series{10}
\corrector{Christian Piske}
\lemmahead{Lemmhead}
\preconditionhead{Prehead}
\assertionhead{Asshead}
\indbasehead{Basehead}
\indhyphead{Hyphead}
\indstephead{Stehead}
\cfoot{page \pagemark{} de \pageref{LastPage}}

\begin{document}

\task{Aufgabe 8.3}

\begin{lemma}
  Je suis un lemme.
\end{lemma}

\begin{precondition*}
Sei $L: \mathbb{N} \rightarrow \mathbb{N}$ mit  
\end{precondition*}
\begin{equation*}
  L(n) = 
  \begin{cases}
  \alpha &\text{falls } n=0\\
  \alpha + \beta n + \max\{L(m-1) + L(n-m) | m \in \{1,...,n\}\} & \text{sonst}
  \end{cases}
\end{equation*}
eine Funktion, die die worst-case Laufzeit von Quicksort angibt.

\begin{assertion*}
Es gilt: $\forall n \in \mathbb{N}_{\geq 0}: L(n) \leq \alpha (2n+1) + \frac{\beta}{2} n (n+1)$
\end{assertion*}

\begin{proof} Im Folgenden verwenden wir $A(n)$ als Abkürzung für 
\begin{equation*}
\forall n \in \mathbb{N}_{\geq 0}: L(n) \leq \alpha (2n+1) + \frac{\beta}{2} n (n+1)    
\end{equation*}

\begin{indbase*} Wir zeigen, dass $A(0)$ gilt.
\begin{align*}
& \phantom{{}={}} L(0) & \text{Def. } L \\  & = \alpha & \\
& = \alpha \cdot (2 \cdot 0 + 1) + \frac{\beta}{2} \cdot 0 \cdot (0+1)& \\
\end{align*}
Also gilt insbesondere $L(0) \leq \alpha \cdot (2 \cdot 0 + 1) + \frac{\beta}{2} \cdot 0 \cdot (0+1)$. Also gilt $A(0)$.
\end{indbase*}

\begin{indhyp*}  
Sei $n \in \mathbb{N}_{\geq 0}$ so gegeben, dass $\forall m \in \mathbb{N}_{\geq 0}: m \leq n \Rightarrow A(m)$ gilt.
\end{indhyp*}

\begin{indstep*}
Bevor wir zeigen, dass $A(n+1)$ gilt, sind zwei Vorüberlegungen zur Größe von $L(m-1)$ und $L(n+1-m)$ für $m \in \{1, ..., n+1\}$ notwendig.
Wegen $m \leq n+1$ gilt $m-1 \leq (n+1) - 1 = n$. Also gilt wegen der IV:
\begin{align*}
& \phantom{{}={}} L(m-1) & \text{IV} \\
& \leq \alpha (2(m-1)+1) + \frac{\beta}{2} (m-1) (m-1+1) & \\
& =    \alpha (2m-1) + \frac{\beta}{2} m(m-1) & \tag{i}\\
\end{align*}

Wegen $m \geq 1$ gilt außerdem $(n+1)-m \leq (n+1) - 1 = n$. Also gilt wegen der IV:
\begin{align*}
& \phantom{{}={}} L(n+1-m) & \text{IV} \\
& \leq \alpha (2(n+1-m)+1) + \frac{\beta}{2} (n+1-m) (n+1-m+1) & \\
& = \alpha (2n-2m+3) + \frac{\beta}{2} (n-m+1) (n-m+2) & \\
& = \alpha (2n-2m+3) + \frac{\beta}{2} (n^2 -nm +2n -nm +m^2 -2m +n-m+2) & \\
& = \alpha (2n-2m+3) + \frac{\beta}{2} (m^2 -2nm -3m +n^2 +3n +2) & \\
& = \alpha (2n-2m+3) + \frac{\beta}{2} (m(m-2n-3) +n^2 +3n +2) & \tag{ii}\\
\end{align*}

Also gilt:
\begin{align*}
& \phantom{{}={}} \max\{L(m-1) + L(n+1-m) | m \in \{1,...,n+1\}\} & \text{(i) und (ii)}\\
& \leq \max \{(\alpha (2m-1) + \frac{\beta}{2} m(m-1)) + (\alpha (2n-2m+3) & \\
& \phantom{{}={}} + \frac{\beta}{2} (m(m-2n-3) +n^2 +3n +2)) | m \in \{1,...,n+1\}\} & \\
& = \max \{\alpha ((2m-1)+(2n-2m+3)) + \frac{\beta}{2} (m(m-1) & \\ 
& \phantom{{}={}} + m(m-2n-3) +n^2 +3n +2) | m \in \{1,...,n+1\}\} & \\
& = \max \{\alpha (2n+2) + \frac{\beta}{2} (m(m-1+m-2n-3) +n^2 +3n +2) & \\ 
& \phantom{{}={}} | m \in \{1,...,n+1\}\} & \\
& = \max \{\alpha (2n+2) + \frac{\beta}{2} (m(2m-2n-4) +n^2 +3n +2) & \\
& \phantom{{}={}} | m \in \{1,...,n+1\}\} & m \leq n+1\\
& \leq \alpha (2n+2) + \frac{\beta}{2} ((n+1)(2(n+1)-2n-4) +n^2 +3n +2) & \\
& = \alpha (2n+2) + \frac{\beta}{2} ((n+1)(-2) +n^2 +3n +2) & \\
& = \alpha (2n+2) + \frac{\beta}{2} (-2n-2 +n^2 +3n +2) & \\
& = \alpha (2n+2) + \frac{\beta}{2} (n^2 +n) & \\
& = \alpha (2n+2) + \frac{\beta}{2} n(n+1) & \tag{iii}\\
\end{align*}

Also gilt:
\begin{align*}
& \phantom{{}={}} L(n+1)   & \text{(iii)} \\
& = \alpha + \beta (n+1) + \max\{L(m-1) + L(n+1-m) | m \in \{1,...,n+1\}\} & \\
& \leq \alpha + \beta (n+1) + \alpha (2n+2) + \frac{\beta}{2} n(n+1) & \\
& = \alpha (2n+3) + \frac{\beta}{2} 2(n+1) + \frac{\beta}{2} n(n+1) & \\
& = \alpha (2n+3) + \frac{\beta}{2} (2(n+1) + n(n+1)) & \\
& = \alpha (2(n+1)+1) + \frac{\beta}{2} (n+1)(2 + n) & \\
& = \alpha (2(n+1)+1) + \frac{\beta}{2} (n+1)((n+1)+1) & \\
\end{align*}
Also gilt $A(n+1)$.
\end{indstep*}

Wegen IA und IS gilt die Behauptung.

\end{proof}

\end{document}

%%% Local Variables:
%%% mode: latex
%%% TeX-master: t
%%% End:

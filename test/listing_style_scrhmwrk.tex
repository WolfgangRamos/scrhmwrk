\documentclass[lang=english, margins=small, listings=true]{scrhmwrk}

\course{ADS – Übungen}
\tutor[Übungsleiterin]{Malin Rau \& Maren Kaluza}
%\tutor[]{Malin Rau \& Maren Kaluza}
% \tutor[Arschgeigen]{Malin Rau \& Maren Kaluza}
\group[Gr]{3 \& 8}
\participant{Wolfgang Ramos \& Ole Numssen}
\series{7}
\corrector{Christian Pisk}

% \newcommand{\tutor}[2][\expandafter{\@tutortitle}]{
%   \renewcommand{\@tutortitle}{#1}
%   \renewcommand{\@tutor}{#2}
% }

\begin{document}

\lstset{language=Pseudocode}

\task{Aufgabe 7.3}

\paragraph{Voraussetzungen}
Seien $l_{1}$ und $l_{2}$ Listen mit Zahlen.

\paragraph{Vereinbarung}
Gemäß der Einführung von Listen in der Vorlesung gehen wird davon aus,
dass eine Liste aus einer Folge von Knoten (wir nennen diesen Typ
\texttt{node}) besteht. Jeder \texttt{node} enthält ein Datenelement
 und einen Zeiger auf den nächsten Knoten der Liste, wobei der Zeiger
 des letzten Elements der Liste auf das \texttt{null}-Element zeigt. In
 unserer Implementierung nehmen wir an, dass über die Operationen \texttt{value($\cdot$)} auf
 dieses Datenelement und über \texttt{next($\cdot$)} auf den Zeiger
 eines Knoten zugegriffen werden kann. Weiterhin enthält eine Liste
 gemäß Vorlesung einen Zeiger \emph{head}, der auf den ersten Knoten
 der Liste zeigt und einen Zeiger \emph{tail}, der auf den letzten
 Knoten der Liste zeigt. In unserer Implementierung nehmen wir an, dass über die Operationen \texttt{head($\cdot$)} auf
 \emph{head} und über \texttt{tail($\cdot$)} auf \emph{tail} zugegriffen werden kann.

\paragraph{Idee} Konkateniere $l_{1}$ und $l_{2}$ zu einer Gesamtliste. Wandle diese Gesamtliste
in ein Feld um. Sortiere dieses Feld mit \texttt{MERGESORT}.
Erzeuge eine Liste, welche die Wert des Felds \emph{ohne} Duplikate
enthält. \autoref{lst:setunion} zeigt eine Implementierung dieses
Algorithmus in Pseudocode.

\paragraph{Anmerkungen zur Implementierung}
In der aktuellen Form verändert die Implemetierung von
\texttt{SETUNION} die Eingabelisten (s. Zeile 5
\autoref{lst:setunion}). Um das zu vermeiden könnte man in Zeile 5 zunächst eine Kopie der Eingabelisten
\texttt{l1} und \texttt{l2} erstellen (das ist in $\mathcal{O}(n)$
möglich) und dann auf diesen operieren.

\begin{lstlisting}[style=scrhmwrk, caption={Algorithmus \texttt{SETUNION}}, label=lst:setunion, frame=single, mathescape]
// compute the union of the elements from two lists
SETUNION(l1, l2) {
  
  // concatenate lists
  next(tail(l1)) = head(l2);

  // create an array from the elements of the concatenated list
  integer n = 0;
  node nod = head(l1);

  // count elements in the concatenated list
  while nod <> null do
    n = n + 1;
    nod = next(nod);
  od

  erzeuge eindimensinonales Feld F mit n Elementen;
  
  // write elements from the concatenated list to F
  nod = head(l1);
  integer i = 0;
  while nod <> null do
    F[i] = value(nod);
    nod = next(nod);
    i = i + 1;
  od;

  // sort elements in F
  MERGESORT(F);
  return NUB(F);
}

// return a list of unique values from an array
NUB(A) {
  integer n = length(A);
  bool tailset = false; // flag to indicate if tail has been set

  erzeuge Liste l;
  head(l) = null;
  tail(l) = null;
  
  // concatenate the elements of the array to a list (we assume that this loop is not entered if n-1 < 1, so A has at least two elements)
  for i = 1 to n-1 do
    if A[i] > A[i-1] then
      erzeuge Knoten k;
      value(k) = A[i];

      // append k to the head of the list
      next(k) = head(l);
      head(l) = k;
      
      // set tail once
      if not(tailset) then
        tail(l) = k;
        tailset = true;
      fi
    fi
  od

  return l;
}
\end{lstlisting}

\paragraph{Laufzeit der Implementierung}
Sei $n$ die Anzahl der Elemente in \emph{beiden} Listen.
Das Erzeugen eines Feld von Elementen aus der Liste ist in
$\mathcal{O}(n)$ möglich, da zunächst die Gesamtzahl der Elemente in der Liste bestimmt wird (das
liegt in $\mathcal{O}(n)$) und dann jedes Element der Gesamtliste in das Feld
eingefügt wird (ebefalls in $\mathcal{O}(n)$).
Die Laufzeit von \texttt{MERGESORT} liegt gemäß Vorlesung in $\mathcal{O}(n \log n)$.
Die Laufzeit von \texttt{NUB} liegt in $\mathcal{O}(n)$, da jedes
Element der sortierten Liste einmal mit seinem Vorgänger verglichen
wird.
Also liegt die Laufzeit von \texttt{SETUNION} in $\mathcal{O}(n \log
n)$.

\end{document}

%%% Local Variables:
%%% mode: latex
%%% TeX-master: t
%%% End:
